\documentclass[12pt]{article}
\pdfpagewidth 8.5in
\pdfpageheight 11.0in
\usepackage{fullpage}
\usepackage{url}
\usepackage{graphicx}
\usepackage{subfigure}
\usepackage{booktabs}
\usepackage{multirow}
\usepackage{rotating}
\usepackage{float}
\usepackage{acronym}
\usepackage{setspace}
\usepackage{amsmath}
\onehalfspacing


\title{Deformable Mirror Demonstration}
\author{Kristin Berry\\
Ashley K. Carlton\\
Zachary J. Casas\\
James R. Clark\\
Vladimir Eremin\\ 
Zaira G. Garate\\ 
Julian Lemus\\
Immanuel David Madukauwa-David\\
Tam Nguyen T. Nguyen\\
Paul Salcido\\
Alexandra E. Wassenberg 
}

\date{\today}

\begin{document}
\maketitle
\newpage

\tableofcontents
\listoffigures
\listoftables


\section*{List of Acronyms}
\begin{acronym}

\acro{DeMi}{Deformable Mirror}

\end{acronym}
\newpage

%%%%%%%%%%%%%%%%%%%%%%%%%%%%%%%%%%%%%%%%%%%%%%%%%%%%%%
\section{Introduction}
		\subsection{Mission Statement}
		\subsection{Motivation}
\section{Mission Overview}
		\subsection{Requirements}
		\subsection{Concept of Operations}
		\subsection{$N^2$ Diagram}
		\subsection{System Level Budgets}
\section{Subsystems}
		\subsection{Payload}
			\subsubsection{Requirements}
			\subsubsection{Trade Studies and Decisions Made}
			\subsubsection{Analysis}
			\subsubsection{Summary of Outputs}
			\subsubsection{Risks}
			\subsubsection{Future Work}
		\subsection{Power}
		\subsection{Communication}
			\subsubsection{Requirements - ZC}

Most of the Communications requirements (see Appendix~\ref{app:requirements}) are standard Communications requirements for operability. Comm-1 is simply stating what the main goal of the Communications subsystem is. Comm-2 is there to ensure that the link with the ground station is good enough to transmit all of the data that we need to. Comm-2.1 is there so that if there is an error in the transmission, the system will be able to realize this and ask for a correction. Comm-2.2 is so that it doesn’t try to downlink more data than possible because if it does, then data will be lost. Comm-3 is simply there so that the Communications subsystem can ensure access with the ground station. Comm-3.1 and Comm-3.2 are specifics when it comes to communicating with the ground station. Comm-4 is there because if we do not encrypt it, another satellite may be able to receive our commands and learn about the DeMi system. Comm-5 is there to ensure that if there are errors, there will be very few of them. Comm-6 is just to ensure that DeMi follows all regulations set by different government agencies.

			\subsubsection{Trade Studies - ZC}\label{sec:comm_tradestudies}
The main factor in choosing what transceiver to use in our system is the data downlink amount. Because payload is generating so much data, Communications will need to have a high downlink rate to ensure that all of that data can be sent to the ground. Several options were looked into, and shown in Table~\ref{table:comm_transceivers}.

\begin{table}[ht]\label{table:comm_transceivers}
\caption{Communications Trade Study for Maximum Downlink Data Rate}
\begin{center}
    \begin{tabular}{| c | c |} \hline
    	Tranceiver & Maximum Downlink Rate \\ \hline \hline
Espace Payload Telemetry System & $1\ Mb/s$ \\
AstroDev Li-1 UHF Transceiver & $38.4\ kb/s$ \\
ISIS TXS Small Satellite  S-Band Transmitter & $100\ kb/s$ \\
Tyvak UHF Transceiver & $200\ kb/s$ \\
L-3 Cadet Nanosat UHF Radio & $1.5\ Mb/s$ \\
Microhard MHX2420 Modem & $230.4\ kb/s$ \\ \hline 
    \end{tabular}
\end{center}
\end{table}

			\subsubsection{Decisions Made - ZC}

Given the information in Section~\ref{sec:comm_tradestudies}, the L-3 Cadet Nanosat UHF Radio was chosen for use on DeMi. It will transmit at the frequencies and data rates outlined in the requirements, which are $445\ -\ 455\ MHz$ for uplink at $19.2\ kb/s$, $460\ -\ 470\ MHz$ for downlink at $1.5\ Mb/s$. The transceiver also has an extra $4\ GB$ of storage that avionics will be able to use in addition to any of its storage. 

\begin{figure}[ht]\label{fig:comm_Cadet}
\centering
  \includegraphics[width=3in]{commCadet.png}
\caption{Photo of the custom-designed high-speed L3 Cadet radio \cite{DICE}.}
\end{figure}

Because the transceiver will be transmitting and receiving at UHF frequencies, a very common antenna choice is to use a measuring tape antenna that is tuned to the frequency that is desired. Because the antenna works like a quarter-wave monopole, one can determine the desired length of the antenna by dividing the wavelength of the transmitted and received signal by 4. Because the transmitting and receiving frequencies are different, the average of these frequencies was used to compute the wavelength. 

\begin{equation}\label{eq:comm_lambda}
\lambda = c/f
\end{equation}

Eq.~\ref{eq:comm_lambda} is used to calculate the wavelength. $\lambda$ is the wavelength, $c$ is the speed of light, and $f$ is the frequency. Using the average frequency of $465\ MHz$, the wavelength equals $0.645\ m$. Then to determine the length of the antenna one can just use the following equation.

\begin{equation}\label{eq:comm_length}
l = \lambda/4
\end{equation}

In Eq.~\ref{eq:comm_length}, $l$ is the length of the antenna, and $\lambda$ is the wavelength. Using a wavelength of $0.645\ m$, one can find that the desired length of the antenna is about $0.164\ m$. The gain of this antenna will be approximately $3\ dB$ and the beam width approximately $65^\circ$.

\begin{figure}[ht]\label{fig:comm_tape}
\centering
  \includegraphics[width=3in]{commTape.png}
\caption{Photo of stored and deployed measuring tape antenna. The pictured CubeSat has four antennas. DeMi will only be using one \cite{antenna}.}
\end{figure}

The ground station that will be used with this system will be the NASA Wallops UHF Ground Station. This was chosen because the transceiver that will be used was also used in the Dynamic Ionosphere CubeSat Experiment, and the ground station that was used in that mission was the NASA Wallops UHF Ground Station. The diameter if the dish is $18.29\ m$, the gain is $35\ dB$ and the beam width is $2.9^\circ$.

			\subsubsection{Analysis - ZC}

One can calculate the maximum amount of data downlink per pass-by by multiplying the downlink data rate and the access duration per pass-by. Given the downlink data rate of $1.5\ Mb/s$, and the access duration of $586\ s$ per pass-by, the maximum amount of data that can be downlinked in one pass-by is $109.9\ MB$. To account for the fact that the data-rate won’t always be able to be at its maximum, and also for space for telemetry data and error correcting code, we have a margin of a factor of 2, so the maximum amount of data from payload that can be downlinked per pass-by is about $54.95\ MB$. Although this is not as much data as payload is creating, there will be on-board processing so that the amount of data that needs to be downlinked is below the maximum amount that is allotted for payload.

A Link Budget was created in order to determine the Link Margin for a worst case and a best case for uplink and downlink. Appendix~\ref{app:link_budgets} detail these budgets.

The first calculation in the Appendix~\ref{app:link_budgets} is to calculate the worst case propagation path length. We know that the best case length is $500\ km$ because that is the altitude at which the satellite is orbiting. Figure~\ref{fig:comm_angles} shows how a wave propagates through that atmosphere and will be explained beneath.

\begin{figure}[ht]\label{fig:comm_angles}
\centering
  \includegraphics[width=4in]{commAngles.png}
\caption{Diagram of propagation path of a signal from a satellite \cite{ITU-R}.}
\end{figure}

In order to determine the propagation path length, $r$, one must determine the path length of the first layer and the second layer and add them together. The equation for calculating \cite{ITU-R} the path length of a layer is as follows.

\begin{equation}\label{eq:path_length}
a_n = -r_n\cos(\beta_n) + \frac{1}{2}\sqrt{4r_n^2\cos^2(\beta_n)+8r_n\delta_n+4\delta_n^2} 
\end{equation}

In Eq.~\ref{eq:path_length}, $a_n$ is the path length through layer $n$, $r_n$ is the radii from the center of the Earth to the beginning of layer $n$, $\beta_n$ is the exiting incidence angle, and $\delta_n$ is the thickness of layer $n$.
The angle $\beta_n$ can be calculated \cite{ITU-R} using the following equation.

\begin{equation}\label{eq:betan1}
\beta_{n+1} = \arcsin\biggl(\frac{n_n}{n_{n+1}}\sin(a_n)\biggr) 
\end{equation}

In Eq.~\ref{eq:betan1}, $n_n$ is the refractive index of layer $n$, and $\alpha_n$ is the entry incidence angle. Angle $\alpha_n$ can be calculated with the last equation needed in order to calculate \cite{ITU-R} the propagation length.

\begin{equation}\label{eq:a_n}
a_n = \pi - \arccos \biggl(\frac{-a_n^2 - 2r_n\delta_n - \delta_n^2}{2a_n r_n + 2a_n \delta_n}\biggr) 
\end{equation}

Values that still need to be defined are $\beta_1$ (because it cannot be calculated using Eq.~\ref{eq:a_n}), $\delta_1$, $\delta_2$, $n_1$, and $n_2$. $\beta_1$ is just equal to the complimentary angle to the minimum angle at which the Wallops Ground Station can track, which is $5^\circ$. $\delta_1$ will be equal to the altitude at which the atmosphere ends which one can say is about $100\ km$. $\delta_2$ is just the orbiting altitude minus $\delta_1$. $n_1$, which is simple the refractive index of air, is $1.000293$. The last remaining value to be defined is $n_2$, which is the refractive index of a vacuum, which is just 1. Using these equations, one can find that $a_1$ equals $707\ km$, and $a_2$ equals $1375\ km$. This means that the worst case propagation path length is equal to $2082\ km$.

The next value calculated in the Link Budgets is the Equivalent, Isotropic Radiated Power (EIRP). In order to calculate the EIRP \cite{SMAD}, the following equation should be used.

\begin{equation}\label{eq:EIRP}
EIRP = P_{tx} + G_{tx} - L_{output} 
\end{equation}

In Eq.~\ref{eq:EIRP}, $P_{tx}$ is the transmitter power, $G_{tx}$ is the transmit antenna gain, and $L_{output}$ is the output loss which is equal to all losses associated with the transmitter. All of these values can be found in the Link Budget (Appendix~\ref{app:link_budgets}).

Another calculated value in the Link Budget is the Space Loss. This is the loss of the signal as it is transmitted through space. It can be calculated \cite{SMAD} using the following equation.

\begin{equation}\label{eq:L_s}
L_s = 92.45 + 20\log_{10}(r) + 20\log_{10}(f) 
\end{equation}

In Eq.~\ref{eq:L_s}, $L_s$ is space loss, $r$ is the propagation path length, and $f$ is the frequency of the transmitted signal. All of these values can also be found in the Link Budget (Appendix~\ref{app:link_budgets}.

The next equation can be used to calculate \cite{pozar} the transmitting antenna temperature.

\begin{equation}\label{eq:t_antenna}
T_{antenna} = \eta T_{sky} + (1 + \eta) \frac{T_{sky} + T_{ground}}{2} 
\end{equation}

In Eq.~\ref{eq:t_antenna}, $\eta$ is the antenna efficiency, $T_{sky}$ is the temperature of the space behind the receiving antenna. The temperatures used for uplink are determined by using the background temperature of space. For downlink, the temperature of the Earth is used. $T_{ground}$ is the temperature of the ground around the antenna. For uplink, this is the temperature of the Earth, and for Downlink, it is the background temperature of space.

The system noise temperature, $T_{sys}$, is calculated by adding the transmitting antenna temperature and the receiver temperature \cite{pozar}. The receiver temperature for both cases is unique to the receiver and can be determined by the manufacturer.

The next calculated value is the Receiver Gain to Noise Temperature. This is computed \cite{SMAD} using the following equation.

\begin{equation}\label{eq:G/T}
\frac{G}{T} =  G_{rx} - T_{sys} 
\end{equation}

In Eq.~\ref{eq:G/T}, $\frac{G}{T}$ is the Receiver Gain to Noise Temperature, $G_{rx}$ is the receiving antenna gain, and $T_{sys}$ is the system noise temperature. Using $\frac{G}{T}$, the total losses in the system, and EIRP one can determine \cite{SMAD} the Receiver Carrier to Noise Ratio.

\begin{equation}\label{eq:c/n}
\frac{C}{N_0} = EIRP + \frac{G}{T} - L_{total} + 228.6 
\end{equation}

Using that value and the data rate in decibels, one can calculate \cite{SMAD} the Energy per bit to Noise Ratio.

\begin{equation}\label{eq:eb/n}
\frac{E_b}{N_0} = \frac{C}{N_0} - R_b 
\end{equation}

Finally, the link margin can be determined by subtracting the required Energy per bit to Noise Ratio from the predicted one. 

\begin{figure}[ht]\label{fig:comm_best}
\centering
  \includegraphics[width=4in]{commBest.png}
\caption{Representation of Satellite and Ground Station during Best Case Communication.}
\end{figure}

\begin{figure}[ht]\label{fig:comm_worst}
\centering
  \includegraphics[width=4in]{commWorst.png}
\caption{Representation of Satellite and Ground Station during Worst Case Communication.}
\end{figure}

			\subsubsection{Summary of Outputs - ZC}

There were several outputs during the design process from the Communications subsystem. These outputs include the Ground station to Orbits, which is NASA Wallops UHF Ground Station, the beam width of Wallops UHF Ground Station and DeMi to ADCS, which are $2.9^\circ$ and $65^\circ$ respectively, the operational and survival temperatures of the transceiver to Thermal, which are $-20^\circ$ C to $70^\circ$ C and $-40^\circ$ C to $80^\circ$ C respectively. The power for each mode to Power, and the mass and volume to Structures are detailed in Table~\ref{table:comm_summary_outputs}

\begin{table}[ht]\label{table:comm_summary_outputs}
\caption{Communications Budgets}
\begin{center}
    \begin{tabular}{|c||c|} \hline
    	Output & Value \\ \hline \hline
    Power (Standby) & $0.51\ W$  \\
    Power (Data Capture) & $0.51\ W$ \\
    Power (Downlink) & $11.44\ W$ \\
    Power (Uplink) & $0.51\ W$ \\
    Power (Safe Mode) & $0.51\ W$ \\
    Mass & $0.235\ kg$  \\
    Volume & $0.069\ U$ \\ \hline 
    \end{tabular}
\end{center}
\end{table}

			\subsubsection{Risks - ZC}

As of PDR, there were no risks for the Communications subsystem. Towards the beginning of the design process, there were huge concerns about Communications not being able to downlink enough of the data from Payload, but since then, Avionics has decided that DeMi will be doing on-board processing so that all of the data does not need to be downlinked.

			\subsubsection{Future Work - ZC}

Moving on from here, the main thing that should be done is research and analysis of measuring tape antennas. There is currently not very much information out there that is available to DeMi concerning them, so research to determine the exact gain, beam width and radiation patterns should be conducted.		
		
		\subsection{Avionics}
			\subsubsection{Requirements - VE}
The Avionics subsystem requirements are presented in Appendix~\ref{app:requirements}.  The main driving requirements are number 3 and 2 in the order of importance because they are the most challenging to satisfy with a current level of technology.

			\subsubsection{Trade Studies - VE}
We have two main options that are able to satisfy the requirements for Avionics subsystem. Both are shown in Table~\ref{table:avionics_hardware_options}

\begin{table}[ht]\label{table:avionics_hardware_options}
\caption{Hardware Options}
\begin{center}
    \begin{tabular}{| c || p{6cm} | p{6cm} |} \hline
     &	Processing on-board & Raw images to the ground \\ \hline \hline
    Component Name & The Steepest Ascent Mission Interface Computer CS-MIC-G-EM & Single Board Computer Motherboard + PPM with TI MSP430F2618 \\ \hline
    Power Consumption & $0.5\ -\ 1.25$ W & $10$ mW \\ \hline
    Capabilities & Telemetry/Telecommand + Real time image processing on FPGA & Telemetry/Telecommand\\ \hline
    Storage Capacity & Up to 16 GB & Up to 2 GB \\ \hline
    Processors & TI MSP430  + Xilinx FPGA (model can be selected) & TI MSP430F2618 \\ \hline
    Interfaces & I2C, SPI, UART & I2C, SPI, UART \\ \hline
    Mass & $62$ g & $88\ -\ 114$ g \\ \hline 
    \end{tabular}
\end{center}
\end{table}

Since we have identified the Payload driven requirements as the most challenging, we have to start selection of appropriate hardware from that.

\begin{table}[ht]\label{table:avionics_modes}
\caption{Payload image capturing modes}
\begin{center}
    \begin{tabular}{| c || c | c | c |} \hline
    	Mode & 640x480 px subframe &  640x480 px subframe & 1280x1024 px  full frame \\ \hline \hline
    Frame Rate & $100\ fps$ & $10\ fps$ & $10\ fps$ \\
    Data Rate & $310\ Mbit/s$ & $31\ Mbit/s$ & $131\ Mbit/s$ \\
    Duration & $30\ s$ & $300\ s$ & $60\ s$ \\
    Memory Required & $1.14\ GB$ & $1.14\ GB$ & $0.96\ GB$ \\ \hline 
    \end{tabular}
\end{center}
\end{table}

Let’s consider approach when we downlink all the data generated by the Payload without processing it. From Table~\ref{table:avionics_modes} we can see that Payload will be generating around $1\ GB$ of data every time it is run. For a given $1.5\ Mbit/s$ downlink speed in this case we will need $600\ s$ to downlink all the data captured or 10 ground accesses. Since we have only one ground access every three orbits in general this particular approach seems to be too ineffective in terms of Payload active time. 

That’s why we mainly consider the second approach when we do all necessary calculations onboard and send only results to the ground. It dramatically reduces the amount of transferred data to around $10\ -\ 100\ KB$ instead of gigabytes and enables to implement a closed loop deformable mirror control system.

After we have solved a problem with data storage we can focus on the core of every Avionics subsystem - its processor. Now we have higher but still reasonable requirements for processing power according to the computation tasks that it will be solving: 1. Centroid, delta x and delta y, slope reconstruction, and 2. Linear algebra for mirror controller.
The Steepest Ascent Mission Interface Computer CS-MIC-G-EM is a good fit for such tasks because it has an FPGA to be configured for image processing and a microcontroller for general tasks such as telemetry and ADCS computations.

\subsubsection{Decisions Made - VE}

We have made a decision to use CS-MIC-G-EM (Figure~\ref{fig:avionics_MIC}) mainly because it enables onboard image processing which reduces the amount of data we send to the ground and provides with a capability to build a closed loop deformable mirror control system.

\begin{figure}[ht]\label{fig:avionics_MIC}
\centering
  \includegraphics[width=4in]{avionicsMIC.jpg}
\caption{Mission Interface Computer CS-MIC-G-EM \cite{avionics_clyde_space}}
\end{figure}

It is also capable of providing the following interfaces with other subsystems see Table~\ref{table:avionics_interfaces}.

\begin{table}[ht]\label{table:avionics_interfaces}
\caption{Avionics hardware interfaces with other subsystems (*assuming update 10 times per second).}
\begin{center}
    \begin{tabular}{| c | c | c | c |} \hline
    	Subsytem & Component & Interface & Data rate \\ \hline \hline
    Payload & Detector (IDS UI-5241LE-M) & GbE & $550\ Mbit/s$ max  \\
     & Mirror driver (BMC Mini-Driver) & USB 2.0 & $480\ Mbit/s$ max \\
     & Laser (ThorLabs CPS 186) & GPIO & -- \\ \hline
    Power & EPS, PDM & 12C & $400\ Kbit/s$ max \\ \hline
    ADCS & 5 sun sensors & analog & $100\ bit/s$* \\
     & ADIS16305 IMU/Magnetometer & SPI & $160\ bit/s$* \\
     & Torque coils & 12C (via PDM) & -- \\ \hline
    Thermal & 14 Temperature Sensors & analog & $100\ bit/s$* \\
     & Thermal Heater & 12C (via PDM) & -- \\ \hline
    Communication & Cadet NanoSat UHF Radio & RS232 & $1.5\ Mbit/s$ \\ \hline 
    \end{tabular}
\end{center}
\end{table}

			\subsubsection{Analysis - VE}
The system architecture in general is shown in Figure~\ref{fig:avionics_architecture}. The Mission Interface Computer we have selected allows us to separate telemetry from image processing data. The first is processed on CPU while the latter is done on FPGA.

\begin{figure}[ht]\label{fig:avionics_architecture}
\centering
  \includegraphics[width=7in]{avionicsarchitecture.jpg}
\caption{Caption goes here...}
\end{figure}

			\subsubsection{Summary of Outputs - ZC}
The Avionics system had very few outputs to the other subsystems. It had to give values to Thermal for operational temperature range for the computer which is $-25^\circ$C to $85^\circ$C. The survival temperature is unknown. It also had to give outputs to power for power consumed during the different modes, and to structures for the mass and volume. These outputs, or budgets are detailed in Table~\ref{table:avionics_summary_outputs}.

\begin{table}[ht]\label{table:avionics_summary_outputs}
\caption{This is a summary of the mass totals, power needs, and thermal needs of the avionics system.}
\begin{center}
    \begin{tabular}{|c||c|} \hline
    	Output & Value \\ \hline \hline
    Power (Standby) & $0.5\ W$  \\
    Power (Data Capture) & $1.25\ W$ \\
    Power (Downlink/Uplink) & $0.5\ W$ \\
    Power (Safe Mode) & $0.5\ W$ \\
    Mass & $0.062\ kg$  \\
    Volume & $0.104\ U$ \\ \hline 
    \end{tabular}
\end{center}
\end{table}

			\subsubsection{Risks - ZC}
There are two risks that avionics faces currently. One risk which is a high consequence, but a very low likelihood is that the system will not be capable of processing all of the incoming data for payload. Payload is outputting a lot of data, and because the design of the code that will do the process has not yet been done, it is unclear how quickly Avionics will be able to process incoming data. The other risk that Avionics is currently faces, which is of very low likelihood, but very high consequence, is that it will not be capable of providing the required latency for Payload or for ADCS. Research needs to be conducted to determine how fast avionics can receive and issue orders to ensure that the computer is fast enough for the system. If it is not fast enough there could be major failures in the system because the attitude of the craft is not correct.

			\subsubsection{Future Work - ZC}
The future work that needs to be conducted is that there should be work done to determine a new solution to interfacing with Payload. The mirror driver interfaces with USB 2.0, however the computer is not able to interface with this. So work should be done into determining how to resolve this. One other thing that should be looked into is ensuring that the system can run all of the necessary calculations at the required speed so that all of the payload data can be processed and sent to the ground. If all of the data can’t be processed, it would be difficult to complete our science goals.

		\subsection{Attitude Determination and Control System}
		\subsection{Thermal}
		\subsection{Structure}
\section{Conclusion}
		\subsection{Risk Analysis}
		\subsection{Future Work}
\section{Acknowledgment}
	
	
%%%%%%%%%%%%%%%%%%%%%%%%%%%%%%%%%%%%%%%%%%%%%%%%%%%%%%
% APPENDICES

\newpage
\appendix
\section{\\Requirements} \label{app:requirements}
Requirements go here.

\newpage
\section{\\Link Budgets} \label{app:link_budgets}

\begin{figure}[ht]
\centering 
\caption{Uplink Budget}
\includegraphics[width=8in]{Uplink_Budget.pdf}
\end{figure}

\begin{figure}[ht]
\centering 
\caption{Downlink Budget}
\includegraphics[width=8in]{Downlink_Budget.pdf}
\end{figure}


\newpage

%%%%%%%%%%%%%%%%%%%%%%%%%%%%%%%%%%%%%%%%%%%%%%%%%%%%%%
% REFERENCES
\begin{thebibliography}{9}

%%%%%%%%%%%%%%%%%%%%%%%%%%%%%%%%%%%%%%%%%%%%%%%%%%%%%%
% REFERENCES

\bibitem{avionics_clyde_space}
Clyde Space. CubeSat Shop. [Online]. \url{http://www.clyde-space.com/cubesat_shop/obdh/364_mission-interface-computer-grande-em}, visited May 5, 2013. 
 
\bibitem{DICE}
“DICE (Dynamic Ionosphere CubeSat Experiment), DICE-1 and DICE-2”, eoPortal, \url{https://directory.eoportal.org/web/eoportal/satellite-missions/d/dice}, Accessed 4/2013.

\bibitem{ITU-R}
“ITU-R P.676-9”, \url{http://www.itu.int/dms_pubrec/itu-r/rec/p/R-REC-P.676-9-201202-I!!PDF-E.pdf}, Accessed 4/2013.

\bibitem{antenna}
“JPL LMRST Antenna Deployment Test”, \url{http://www.youtube.com/watch?v=DSvHKzM8scc}, Accessed 4/2013.

\bibitem{kim00}
   Kim. (2000).
  \emph{\ Simulation Study of A Low-Low Satellite-to-Satellite Tracking Mission}. (Doctoral dissertation)
  The University of Texas at Austin, TX.

\bibitem{SMAD}
Wertz, James, \emph{Space Mission Engineering: The New SMAD}, Microcosm Press, Hawthorne, CA, 2011.

\bibitem{pozar}
Pozar, David M., \emph{Microwave and RF Design of Wireless Systems}, Wiley, New York City, NY, 2001.

\end{thebibliography}

\end{document}

